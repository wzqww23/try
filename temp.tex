\documentclass{article} \usepackage{amsmath} \usepackage{xcolor} \begin{document} \textbf{Student Answer:} \begin{align*} 1: & x^2+y^2-4x+6y-77=0 \\ 2: & (x-2)^2+(y+3)^2=8^2 \\ 3: & Q,Q(2,-3) \\ 4: & AQ:y=-\frac{1}{3}x-\frac{7}{3} \\ 5: & BQ:y=\frac{1}{3}x-\frac{11}{3} \\ 6: & A:y=3x+21 \\ 7: & B:y=-3x+33 \\ 8: & P \\ 9: & (x_p,y_p) \\ 10: & \{y_p=3x_p+21 \\ 11: & y_p=-3x_p+33 \\ 12: & \therefore 3x_p+21=-3x_p+33 \\ 13: & \{x_p=2 \\ 14: & y_p=27 \end{align*} \textbf{Feedback and Grade:} \begin{itemize} \item[Mark 1] \textcolor{green}{Well done for finding the center of the circle, point C, as (2,-3) in step 3.} \item[Mark 2] \textcolor{green}{Good job for showing the correct method for finding the gradient of the line AC as -1/3 in step 4 and the gradient of the line BC as 1/3 in step 5.} \item[Mark 3] \textcolor{green}{Great work for finding the gradient of the tangent at A as 3 in step 6 and the gradient of the tangent at B as -3 in step 7.} \item[Mark 4] \textcolor{green}{Well done for writing the equations of the tangents as y=3x+21 in step 6 and y=-3x+33 in step 7.} \item[Mark 5] \textcolor{green}{Good job for setting up the equation to find the point of intersection of the tangent lines in step 12.} \item[Mark 6] \textcolor{green}{Excellent work for finding the coordinates of the point of intersection of the tangent lines as (2,27) in steps 13 and 14.} \end{itemize} \textbf{Score: 6/6} \textbf{Comments:} Great job! You have successfully found the point of intersection of the tangent lines to the circle at points A and B. Keep up the good work! \end{document}