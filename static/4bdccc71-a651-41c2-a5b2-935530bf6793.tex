\documentclass{article}
\usepackage{amsmath}
\usepackage{xcolor}

\begin{document}

\textbf{Student Answer:}
\begin{align*}
1: & 488-(475-321+259) \\
2: & =488-(154+259) \\
3: & =488-413 \\
4: & =75 \\
5: & 43.9-(46.7-24.7+2.05) \\
6: & =43.9-(22+2.05) \\
7: & =43.9-24.05 \\
8: & =19.85 \\
9: & (7643+1257)-:20-258 \\
10: & =8900-:20-258 \\
11: & =445-258 \\
12: & \approx 187 \\
13: & 2400+300-:6-123 \\
14: & =2400+50-123 \\
15: & =2450-123 \\
16: & =2327 \\
\end{align*}

\textbf{Feedback and Grade:}
\begin{itemize}
\item[Mark 1] \textcolor{red}{Not gained. The student did not write the center of the circle as (2,-3).}
\item[Mark 2] \textcolor{red}{Not gained. The student did not show a correct method for finding the gradient of the line AC.}
\item[Mark 3] \textcolor{red}{Not gained. The student did not find the gradient of the tangent at A or B, nor did they find the equations of the tangents.}
\item[Mark 4] \textcolor{red}{Not gained. The student did not find the coordinates of the point of intersection of the tangent lines.}
\end{itemize}

\textbf{Score: 0/6}

\textbf{Comments:} It seems that you have not attempted to solve the given problem. Please review the question and try again. Remember to find the center of the circle, the gradients of the lines connecting the center to points A and B, and the equations of the tangent lines at points A and B. Finally, find the point of intersection of these tangent lines.