\documentclass{article}
\usepackage{amsmath}
\usepackage{xcolor}

\begin{document}

\textbf{Student Answer:}
\begin{align*}
1: & y=0 \\
2: & x_1=11 \\
3: & x_2=-7 \\
4: & x^{2}-4x-77=0 \\
5: & (b\pm\sqrt{b^{2}-4ac})/(2a)a-1b=-4 \\
6: & (=-77 \\
7: & =11 \\
8: & -7 \\
9: & 2x+2y(dy)/(dx)-4+6(dy)/(dx)=0. \\
10: & (dy)/(dx)(2y+6)+2x-4=0. \\
11: & (dx)/(dx)=(4-2x)/(2y+6) \\
12: & a:(4-2z)/(6)=-3 \\
13: & b:(4+14)/(6)=3
\end{align*}

\textbf{Feedback and Grade:}
\begin{itemize}
\item[Mark 1] \textcolor{red}{The center of the circle is not found in the student's answer.}
\item[Mark 2] \textcolor{red}{The gradient of the line AC is not calculated in the student's answer.}
\item[Mark 3] \textcolor{red}{The gradient of the tangent at A or B is not found in the student's answer.}
\item[Mark 4] \textcolor{red}{The equations of the tangents are not found in the student's answer.}
\item[Mark 5] \textcolor{red}{The point of intersection of the tangent lines is not calculated in the student's answer.}
\item[Mark 6] \textcolor{red}{The coordinates of the point of intersection of the tangent lines are not found in the student's answer.}
\end{itemize}

\textbf{Score: 0/6}

\textbf{Comments:} Your work does not address the problem of finding the point of intersection of the tangent lines to the circle at points A and B. It is important to first find the center of the circle and then use the gradients of the lines connecting the center to points A and B to find the gradients of the tangent lines. From there, you can find the equations of the tangent lines and solve for their point of intersection. Please review the concepts of circle equations, gradients, and tangent lines, and try the problem again.

\end{document}