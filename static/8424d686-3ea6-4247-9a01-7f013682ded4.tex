\documentclass{article}
\usepackage{amsmath}
\usepackage{xcolor}

\begin{document}

\textbf{Student Answer:}
\begin{align*}
1: & (x^{2}-4x+4)+(y^{2}+6y+9)-13-77=0 \\
2: & (x-2)^{2}+(y+3)^{2}=90 \\
3: & C: (2, -3) \\
4: & OA, OB \\
5: & y=kx+c \\
6: & -3=2k+c \\
7: & 0=-7k+c \\
8: & k=-\frac{1}{3} \\
9: & k=3 \\
10: & c=21 \\
11: & y=kx+c \\
12: & -3=2k+c \\
13: & 0=11k+c \\
14: & k=\frac{1}{3} \\
15: & k=-3 \\
16: & c=33 \\
17: & y=3x+21 \\
18: & y=-3x+33 \\
19: & 0=0=0=-6x+12 \Rightarrow x=2,y=27 \\
\end{align*}

\textbf{Feedback and Grade:}
\begin{itemize}
\item[Mark 1] \textcolor{green}{Well done for finding the center of the circle, point C, as (2,-3) in step 3.}
\item[Mark 2] \textcolor{green}{Good job for showing the correct method for finding the gradient of the line AC as -1/3 in step 8.}
\item[Mark 3] \textcolor{green}{Great work for finding the gradient of the tangent at A as 3 in step 9.}
\item[Mark 4] \textcolor{green}{Nice job for finding the equation of the tangent at A as y=3x+21 in step 17.}
\item[Mark 5] \textcolor{green}{Well done for finding the gradient of the tangent at B as -3 in step 15.}
\item[Mark 6] \textcolor{green}{Excellent work for finding the coordinates of the point of intersection of the tangent lines as (2,27) in step 19.}
\end{itemize}

\textbf{Score: 6/6}

\textbf{Comments:} Great job! You have successfully found the point of intersection of the tangent lines to the circle at points A and B. Keep up the good work!

\end{document}