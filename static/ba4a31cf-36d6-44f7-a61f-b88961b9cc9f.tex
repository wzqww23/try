\documentclass{article}
\usepackage{amsmath}
\usepackage{xcolor}

\begin{document}

\textbf{Student Answer:}
\begin{align*}
Step 1: & (x^{2}-4x+4)-4+(y^{2}+6y+9)-9-77=0 \\
Step 2: & (x-2)^{2}+(y+3)^{2}=9 \\
Step 3: & (2,-3) \\
Step 4: & h=(11-2)/(0-3))=3 \\
Step 5: & k!=-(1)/(3) \\
Step 6: & (y-0)=-(1)/(3)(x-11) \\
Step 7: & y=-(1)/(3)x+(11)/(3) \\
Step 8: & h_2=(2+7)/(-3-2)=-3 \\
Step 9: & k _|_ k=(1)/(3) \\
Step 10: & (y-0)=(1)/(3)(x+7) \\
Step 11: & y=(1)/(3)x+(7)/(3) \\
Step 12: & (1)/(3)x+(7)/(3)=-(1)/(3)x+(11)/(3) \\
Step 13: & (2)/(3)x=(4)/(3) \\
Step 14: & x=2 \\
\end{align*}

\textbf{Feedback and Grade:}
\begin{itemize}
\item[Mark 1] \textcolor{green}{You correctly identified the center of the circle as (2,-3) in step 3. Well done!}
\item[Mark 2] \textcolor{red}{You did not correctly find the gradient of the line AC. You should have used the formula for the gradient of a line, which is (y2-y1)/(x2-x1).}
\item[Mark 3] \textcolor{green}{You correctly found the gradient of the tangent at A and B in steps 4 and 8. Good job!}
\item[Mark 4] \textcolor{red}{You did not correctly find the equations of the tangents. Remember, the equation of a line is y=mx+c, where m is the gradient and c is the y-intercept.}
\item[Mark 5] \textcolor{red}{You did not correctly find the point of intersection of the tangent lines. The correct method is to equate the two equations of the tangents and solve for x and y.}
\item[Mark 6] \textcolor{red}{You did not correctly find the coordinates of the point of intersection of the tangent lines. The correct coordinates are (2,27).}
\end{itemize}

\textbf{Score: 2/6}

\textbf{Comments:} You've made a good start by correctly identifying the center of the circle and the gradients of the tangents. However, you need to work on finding the equations of the tangents and the point of intersection of these tangents. Remember to use the correct formulas and methods. Keep practicing and you'll get there!

\end{document}