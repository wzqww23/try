\documentclass{article}
\usepackage{amsmath}
\usepackage{xcolor}

\begin{document}

\textbf{Student Answer:}
\begin{align*}
Step 1: & (x^{2}-4x+4)-4+(y^{2}+6y+9)-9-77=0 \\
Step 2: & (x-2)^{2}+(y+3)^{2}=9 \\
Step 3: & (2,-3) \\
Step 4: & h=(11-2)/(h-3)=3 \\
Step 5: & k\neq-(1)/(3) \\
Step 6: & (y-0)=-(1)/(3)(x-11) \\
Step 7: & y=-(1)/(3)x+(11)/(3) \\
Step 8: & h_2=(2+7)/(-3+2)=-3 \\
Step 9: & k \perp k=(1)/(3) \\
Step 10: & (y-0)=(1)/(3)(x+7) \\
Step 11: & y=(1)/(3)x+(7)/(3) \\
Step 12: & (1)/(3)x+(7)/(3)=-(1)/(3)x+(11)/(3) \\
Step 13: & (2)/(3)x=(4)/(3) \\
Step 14: & x=2 \\
\end{align*}

\textbf{Feedback and Grade:}
\begin{itemize}
\item[Mark 1] \textcolor{green}{You correctly identified the center of the circle as (2,-3) in step 3. Well done!}
\item[Mark 2] \textcolor{red}{You did not show a correct method for finding the gradient of the line AC. Please remember to calculate the gradient using the formula (y2-y1)/(x2-x1).}
\item[Mark 3] \textcolor{green}{You correctly found the gradients of the tangents at points A and B in steps 4 and 8. Good job!}
\item[Mark 4] \textcolor{green}{You correctly wrote the equations of the tangents in steps 6 and 10. Keep it up!}
\item[Mark 5] \textcolor{green}{You correctly set the equations of the tangents equal to each other in step 12. Well done!}
\item[Mark 6] \textcolor{red}{You did not find the correct coordinates of the point of intersection of the tangent lines. The y-coordinate should be 27, not 2.}
\end{itemize}

\textbf{Score: 4/6}

\textbf{Comments:} You have a good understanding of the concepts involved in this problem. However, you need to be more careful when calculating the gradient and the coordinates of the point of intersection. Keep practicing and you will improve!

\end{document}