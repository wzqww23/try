\documentclass{article}
\usepackage{amsmath}
\usepackage{xcolor}

\begin{document}

\textbf{Student Answer:}
\begin{align*}
1: & (x^{2}-4x+4)-4+(y^{2}+6y+9)-9-77=0 \\
2: & (x-2)^{2}+(y+3)^{2}=9 \\
3: & (2,-3) \\
4: & h=(11-2)/(h-3)=3 \\
5: & k!=-(1)/(3) \\
6: & (y-0)=-(1)/(3)(x-11) \\
7: & y=-(1)/(3)x+(11)/(3) \\
8: & h_2=(2+7)/(-3+2)=-3 \\
9: & k_\perp k=(1)/(3) \\
10: & (y-0)=(1)/(3)(x+7) \\
11: & y=(1)/(3)x+(7)/(3) \\
12: & (1)/(3)x+(7)/(3)=-(1)/(3)x+(11)/(3) \\
13: & (2)/(3)x=(4)/(3) \\
14: & x=2
\end{align*}

\textbf{Feedback and Grade:}
\begin{itemize}
\item[Mark 1] \textcolor{green}{Well done for finding the center of the circle as (2,-3) in step 3.}
\item[Mark 2] \textcolor{green}{Good job for finding the gradient of the line AC as 3 in step 4.}
\item[Mark 3] \textcolor{green}{Correctly found the gradient of the tangent at A as -1/3 in step 5 and the gradient of the tangent at B as 1/3 in step 9.}
\item[Mark 4] \textcolor{green}{Great work in finding the equations of the tangents at A and B in steps 6 and 10.}
\item[Mark 5] \textcolor{green}{You correctly set up the equation to find the point of intersection in step 12.}
\item[Mark 6] \textcolor{green}{Well done for finding the x-coordinate of the point of intersection as 2 in step 14.}
\end{itemize}

\textbf{Score: 6/6}

\textbf{Comments:} Excellent work! You have shown a clear understanding of the concepts and applied them correctly to find the point of intersection of the tangent lines. Keep up the good work!

\end{document}